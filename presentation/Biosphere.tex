\documentclass{ctexart}
\usepackage{listings}
\usepackage{enumitem}
\begin{document}

\section{Biosphere技术细节}
Musinet的遗传算法部分称作Biosphere(生物圈)。

Biosphere的基本流程与一般GA相同,即将信息编码后放在一起作为种群,
然后依次对这个种群实施基因突变、选择复制、交配得到下一代。

\subsection{信息表示}
Musinet把一个乐谱表示为如下Python数据结构:
\begin{lstlisting}
note = [pitch, dynamic, rhythm, duration]
notes = [note]
part = [instrument, notes, rank]
score = [part]
\end{lstlisting}
GA只使用长度为512的notes,即一个染色体是长度为512的列表,列表元素是长度为4的列表。

\subsection{适应度与自然选择}
GA的适应度函数直接使用打分网络(ValueNet)的输出。其基本考虑是,CNN可能在识别音乐特征上比RNN更有优势。

自然选择的过程使得种群的整体适应度提升,自然选择后种群就为交配做好了准备。
Biosphere使用了简单了轮盘赌算法。流程如下:
\begin{enumerate}[nosep]
  \item 将每个个体的适应度相加得到$T$
  \item 任取$[0,T]$之间的随机数$M$
  \item 令$S$等于前$i$个个体的适应度,且$S$是恰好不小于$M$
  \item 把第$i$个个体放入交配池
\end{enumerate}

\subsection{杂交算子}
Biosphere使用了最简单的单点交换算法,流程如下:
\begin{enumerate}[nosep]
  \item 取种群中$P_c$比例的个体进行交配,其余个体复制到下一代
  \item 对于任意一对个体A、B,任取位点$i$,使得A和B的染色体在该位点前后发生互换
  \item 把得到的两个新染色体放入下一代
\end{enumerate}

\subsection{效果评估}
% TODO

\end{document}
